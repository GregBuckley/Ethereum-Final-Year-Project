\chapter{Introduction}
Over the past thirty years, the number of people and businesses online has grown dramatically. For the internet to be successful for as a means for communication, it is vital to have secure communication where users can be assured of the identity of who they are in contact with.

The two most common ways which this has been achieved is through the use of third party certificate authorities and through the Pretty Good Privacy design which makes use of the Web Of Trust. Unfortunately while these have both been frequently used, both have proven to be faulted with neither being a perfect solution for the future.

The introduction of the Blockchain in the past ten years has sparked interest as a means of creating a distributed decentralised database which is protected from malicious attacks. The rise of successful cryptocurrencies such as Bitcoin has proven that the technology can be used successfully to solve some of the worlds biggest technological problems.

New identity mechanisms have been designed on the Bitcoin Blockchain to take advantage of the decentralised design, but most have had shortcomings due to the fact that the Bitcoin Blockchain was designed initially with only digital currency in mind.

Ethereum has been introduced in the past two years, with the goal to surpass all the existing limitations of the Bitcoin Blockchain with a design made with the intention of the creation of decentralised applications.

This paper looks into the background of the existing identity mechanisms used today, the proposals of using the blockchain as an alternative identity design and most importantly, investigating wether the Ethereum Blockchain has the potential to create a sound solution for the future.
